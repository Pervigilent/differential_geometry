\documentclass{book}
\usepackage{amsfonts}
\usepackage{amsmath}

\begin{document}

	\chapter{Curves}
	
		The \textbf{complex structure} of $\mathbb{R}^2$ is a linear map $J:\mathbb{R}^2\rightarrow\mathbb{R}^2$ defined as follows:
		\begin{equation}
			J(x,y)=(-x,y)\mathrm{.}
		\end{equation} 
		
		Let $f:(a,b)\rightarrow\mathbb{R}^2$ be a curve. Assume that $f$ is defined on a larger interval containing $a$ and $b$. The \textbf{length} of $f$ is given by
		\begin{equation}
			L(f)=\int^a_b{\|f'(t)\|\,dt}\mathrm{.}
		\end{equation}
		
		Let $f:(a,b)\rightarrow\mathbb{R}^2$ be a regular curve, and suppose further that $f$ is given by the parameterization $f(t)=(x(t),y(t))$. The (unsigned) curvature is given by
		\begin{align}
			\kappa(t)&=\frac{f''(t)\cdot Jf'(t)}{\|f'(t)\|^3}\\
			&=\frac{x'y''-y'x''}{(x'^2+y'^2)^{3/2}}\mathrm{.}
		\end{align}
		
	\chapter{Curvature}
		This chapter describes curves in terms of their curvature.
	
		\textbf{Theorem} The absolute value of the curvature and the derivative of arc length of a curve are invariant under Euclidean motions of $\mathbb{R}^2$. The curvature $\kappa$ is preserved by an orientation preserving Euclidean motion of $\mathbb{R}^2$ and changes sign under an orientation reversing Euclidean motion.
		
		\textbf{Theorem} Let $f$ and $g$ be regular curves in $\mathbb{R}^2$ defined on the same interval $(a,b)$. Assume that $f$ and $g$ have the same signed curvature. Then there is an orientation preserving Euclidean motion $E$ of $\mathbb{R}^2$ mapping $f$ into $g$.
		
		\textbf{Theorem} Let $f:(a,b)\rightarrow\mathbb{R}^2$ be a unit speed curve whose curvature is $k:(a,b)\rightarrow\mathbb{R}$. The curve is given by
		\begin{equation*}
			\left\{
			\begin{aligned}
				f(s)&=\left(\int\cos{\theta(s)\,ds}+c_0,\int\sin{\theta(s)\,ds+d_0}\right),\\
				\theta(s)&=\int{k(s)\,ds}+\theta_0,
			\end{aligned}
			\right.
		\end{equation*}
		with constants of integration $c_0, d_0, \theta_0$.
		
	\chapter{Space Curves}
		This chapter highlights space curves.
		
		Let $\gamma:(a,b)\rightarrow\mathbb{R}^n$ be an arc length parametrization of a curve. (We can say that an arc length parametrization of a curve is the same as a unit speed curve?) Then the unit tangent vector $\mathbf{T}$ (called the unit tangent vector field in the text) is given by
		\begin{equation}
			\mathbf{T}(s)=\gamma'(s),\quad s\in(a,b)\mathrm{.}
		\end{equation}
		
		The curvature is the magnitude of the acceleration
		\begin{align}
			\kappa(s)&=\|\mathbf{T}'(s)\|\\
			&=\|\gamma''(s)\|
		\end{align}
		which is consistent with the definition in the previous chapter.
		
		The unit normal vector $\mathbf{N}$ is given by 
		\begin{align}
			\mathbf{N}&=\frac{\mathbf{T}'}{\kappa}\\
			&=\frac{\mathbf{T}'}{\|\mathbf{T}'(s)\|}
		\end{align}
		and is called the principal normal vector field in the text.
		
		The Frenet binormal vector is given by
		\begin{equation}
			\mathbf{B}(s)=\mathbf{T}(s)\times\mathbf{N}(s)
		\end{equation}
		and is called the binormal vector field in the text.
		
		The Frenet-Serret frame is ${\mathbf{T},\mathbf{N},\mathbf{B}}$. It is called the Frenet frame field in the text.
		
		We define the torsion $\tau$ as the speed of rotation of the binormal vector at a given point. The formula for the torsion is then
		\begin{equation}
			\mathbf{B}'=-\tau\mathbf{N}\mathrm{.}
		\end{equation}
		This gives $\tau=-\mathbf{N}\cdot\mathbf{B}'$.
		
		The Frenet-Serret formulas are given by
		\begin{equation}
			\begin{bmatrix}
				\mathbf{T}'\\
				\mathbf{N}'\\
				\mathbf{B}'
			\end{bmatrix}=
			\begin{bmatrix}
				0 & \kappa & 0\\
				-\kappa & 0 & \tau\\
				0 & -\tau & 0
			\end{bmatrix}
			\begin{bmatrix}
				\mathbf{T}\\
				\mathbf{N}\\
				\mathbf{B}
			\end{bmatrix}\mathrm{.}
		\end{equation}

\end{document}
