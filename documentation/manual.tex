\documentclass{book}
\usepackage{amsfonts}
\usepackage{amsmath}

\begin{document}

	\chapter{Curves}
	
		The \textbf{complex structure} of $\mathbb{R}^2$ is a linear map $J:\mathbb{R}^2\rightarrow\mathbb{R}^2$ defined as follows:
		\begin{equation}
			J(x,y)=(-x,y)\mathrm{.}
		\end{equation} 
		
		Let $f:(a,b)\rightarrow\mathbb{R}^2$ be a curve. Assume that $f$ is defined on a larger interval containing $a$ and $b$. The \textbf{length} of $f$ is given by
		\begin{equation}
			L(f)=\int^a_b{\|f'(t)\|\,dt}\mathrm{.}
		\end{equation}
		
		Let $f:(a,b)\rightarrow\mathbb{R}^2$ be a regular curve, and suppose further that $f$ is given by the parameterization $f(t)=(x(t),y(t))$. The (unsigned) curvature is given by
		\begin{align}
			\kappa(t)&=\frac{f''(t)\cdot Jf'(t)}{\|f'(t)\|^3}\\
			&=\frac{x'y''-y'x''}{(x'^2+y'^2)^{3/2}}\mathrm{.}
		\end{align}
		
	\chapter{Curvature}
		This chapter describes curves in terms of their curvature.
	
		\textbf{Theorem} The absolute value of the curvature and the derivative of arc length of a curve are invariant under Euclidean motions of $\mathbb{R}^2$. The curvature $\kappa$ is preserved by an orientation preserving Euclidean motion of $\mathbb{R}^2$ and changes sign under an orientation reversing Euclidean motion.
		
		\textbf{Theorem} Let $f$ and $g$ be regular curves in $\mathbb{R}^2$ defined on the same interval $(a,b)$. Assume that $f$ and $g$ have the same signed curvature. Then there is an orientation preserving Euclidean motion $E$ of $\mathbb{R}^2$ mapping $f$ into $g$.
		
		\textbf{Theorem} Let $f:(a,b)\rightarrow\mathbb{R}^2$ be a unit speed curve whose curvature is $k:(a,b)\rightarrow\mathbb{R}$. The curve is given by
		\begin{equation*}
			\left\{
			\begin{aligned}
				f(s)&=\left(\int\cos{\theta(s)\,ds}+c_0,\int\sin{\theta(s)\,ds+d_0}\right),\\
				\theta(s)&=\int{k(s)\,ds}+\theta_0,
			\end{aligned}
			\right.
		\end{equation*}
		with constants of integration $c_0, d_0, \theta_0$.
		
	\chapter{Space Curves}
		This chapter highlights space curves.
		
		Let $\gamma:(a,b)\rightarrow\mathbb{R}^n$ be an arc length parametrization of a curve. (We can say that an arc length parametrization of a curve is the same as a unit speed curve?) Then the unit tangent vector $\mathbf{T}$ (called the unit tangent vector field in the text) is given by
		\begin{equation}
			\mathbf{T}(s)=\gamma'(s),\quad s\in(a,b)\mathrm{.}
		\end{equation}
		
		The curvature is the magnitude of the acceleration
		\begin{align}
			\kappa(s)&=\|\mathbf{T}'(s)\|\\
			&=\|\gamma''(s)\|
		\end{align}
		which is consistent with the definition in the previous chapter.
		
		The unit normal vector $\mathbf{N}$ is given by 
		\begin{align}
			\mathbf{N}&=\frac{\mathbf{T}'}{\kappa}\\
			&=\frac{\mathbf{T}'}{\|\mathbf{T}'(s)\|}
		\end{align}
		and is called the principal normal vector field in the text.
		
		The Frenet binormal vector is given by
		\begin{equation}
			\mathbf{B}(s)=\mathbf{T}(s)\times\mathbf{N}(s)
		\end{equation}
		and is called the binormal vector field in the text.
		
		The Frenet-Serret frame is ${\mathbf{T},\mathbf{N},\mathbf{B}}$. It is called the Frenet frame field in the text.
		
		We define the torsion $\tau$ as the speed of rotation of the binormal vector at a given point. The formula for the torsion is then
		\begin{equation}
			\mathbf{B}'=-\tau\mathbf{N}\mathrm{.}
		\end{equation}
		This gives $\tau=-\mathbf{N}\cdot\mathbf{B}'$.
		
		The Frenet-Serret formulas are given by
		\begin{equation}
			\begin{bmatrix}
				\mathbf{T}'\\
				\mathbf{N}'\\
				\mathbf{B}'
			\end{bmatrix}=
			\begin{bmatrix}
				0 & \kappa & 0\\
				-\kappa & 0 & \tau\\
				0 & -\tau & 0
			\end{bmatrix}
			\begin{bmatrix}
				\mathbf{T}\\
				\mathbf{N}\\
				\mathbf{B}
			\end{bmatrix}\mathrm{.}
		\end{equation}
		
	\chapter{Calculus}
%		Given a vector $v$ in $\mathbb{R}^n$, the directional derivative at a point $x\in\mathbb{R}^n$ is given by
%		\begin{align}
%			(D_vf)(x)&\equiv\left.\frac{\mathrm{d}}{\mathrm{d}t}[f(x+tv)]\right\rvert_{t=0}&\\
%			&=\sum_{i=1}^{n}{v^i\frac{\partial f}{\partial x^i}(x)},&\forall f\in C^\infty(\mathbb{R}^n)\mathrm{.}
%		\end{align}
		Let $f:\mathbb{R}^n\rightarrow\mathbb{R}$ be a differentiable function and let $\mathbf{v}$ be a vector in $\mathbb{R}^n$. The directional derivative in the $\mathbf{v}$ direction at point $\mathbf{x}\in\mathbb{R}^n$ is given by
		\begin{align}
			\nabla_\mathbf{v}f(x)&\equiv\left.\frac{d}{dt}[f(\mathbf{x}+t\mathbf{v})]\right\rvert_{t=0}\\
			&=\sum_{i=1}^{n}{v^i\frac{\partial f}{\partial x_i}(\mathbf{x})}\mathrm{.}
		\end{align}
		
	\chapter{Differentiable Manifolds}
	
		Let $M$ be a topological space. A \textbf{chart} comprises an open subset $U$ of $M$ and a homeomorphism $\phi$ from $U$ to an an open subset of  A differentiable manifold comprises a second countable Hausdorff space M, together with an open subset of some Euclidean space $\mathbb{R}^n$. (Recall that a homeomorphism is a bijective and continuous function between topological spaces that has a continuous inverse function. [Thus, homeomorphisms are isomorphisms of topological spaces, and two topological spaces which are homeomorphic - they share a homeomorphism - are topologically equivalent.])
		
		Given a topological space $M$, a smooth or analytic \textbf{atlas} comprises a collection of charts $\{\phi_a:U_a\rightarrow\mathbb{R}^n\}_{a\in A}$ such that $\{U_a\}_{a\in A}$ covers $M$, and such that for all $\alpha$ and $\beta$ in $A$, the transition map is $\phi_\alpha\circ\phi_\beta^{-1}$ is smooth or real-analytic, respectively. Similarly, a holomorphic atlas is a collection of charts $\{\phi_a:U_a\rightarrow\mathbb{C}^n\}_{a\in A}$ where the open subsets cover $M$ and the transition maps are holomorphic. We can also define a $C^k$ atlas in a similar manner.
		
		A \textbf{topological manifold} (or $C^0$ manifold) comprises a topological space $M$ together with a $C^0$ atlas on $M$. A \textbf{differentiable manifold} is a Hausdorff and second countable topological space $M$, together with a \emph{maximal differentiable} atlas on $M$. (Recall that a topological space is a Hausdorff space if, given any two distinct points, one can find a neighborhood for each point which is disjoint from the neighborhood of the other point.)
		
		Note that instead of defining a chart, Gray defines a patch. A \emph{patch} on a topological space $M$ comprises a pair $(\phi, U)$ where $U$ is an open subset of $\mathbb{R}^n$ and
		\begin{equation}
			\phi:U\rightarrow\phi(U)\subset M
		\end{equation}
		is a homeomorphism of $U$ into an open set $\phi(U)$ of $M$.

\end{document}
