\documentclass{book}
\usepackage{amsfonts}
\usepackage{amsmath}

\begin{document}

	\chapter{Curves}
	
		The \textbf{complex structure} of $\mathbb{R}^2$ is a linear map $J:\mathbb{R}^2\rightarrow\mathbb{R}^2$ defined as follows:
		\begin{equation}
			J(x,y)=(-x,y)\mathrm{.}
		\end{equation} 
		
		Let $f:(a,b)\rightarrow\mathbb{R}^2$ be a curve. Assume that $f$ is defined on a larger interval containing $a$ and $b$. The \textbf{length} of $f$ is given by
		\begin{equation}
			L(f)=\int^a_b{\|f'(t)\|\,dt}\mathrm{.}
		\end{equation}
		
		Let $f:(a,b)\rightarrow\mathbb{R}^2$ be a regular curve, and suppose further that $f$ is given by the parameterization $f(t)=(x(t),y(t))$. The (unsigned) curvature is given by
		\begin{align}
			\kappa(t)&=\frac{f''(t)\cdot Jf'(t)}{\|f'(t)\|^3}\\
			&=\frac{x'y''-y'x''}{(x'^2+y'^2)^{3/2}}\mathrm{.}
		\end{align}
		
	\chapter{Curvature}
		This chapter describes curves in terms of their curvature.
	
		\textbf{Theorem} The absolute value of the curvature and the derivative of arc length of a curve are invariant under Euclidean motions of $\mathbb{R}^2$. The curvature $\kappa$ is preserved by an orientation preserving Euclidean motion of $\mathbb{R}^2$ and changes sign under an orientation reversing Euclidean motion.
		
		\textbf{Theorem} Let $f$ and $g$ be regular curves in $\mathbb{R}^2$ defined on the same interval $(a,b)$. Assume that $f$ and $g$ have the same signed curvature. Then there is an orientation preserving Euclidean motion $E$ of $\mathbb{R}^2$ mapping $f$ into $g$.
		
		\textbf{Theorem} Let $f:(a,b)\rightarrow\mathbb{R}^2$ be a unit speed curve whose curvature is $k:(a,b)\rightarrow\mathbb{R}$. The curve is given by
		\begin{equation*}
			\left\{
			\begin{aligned}
				f(s)&=\left(\int\cos{\theta(s)\,ds}+c_0,\int\sin{\theta(s)\,ds+d_0}\right),\\
				\theta(s)&=\int{k(s)\,ds}+\theta_0,
			\end{aligned}
			\right.
		\end{equation*}
		with constants of integration $c_0, d_0, \theta_0$.		

\end{document}
